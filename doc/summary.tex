\documentclass[a4paper,12pt]{article}

% polyglossia should go first!
\usepackage{polyglossia} % multi-language support
\setmainlanguage{russian}
\setotherlanguage{english}

\usepackage{amsmath} % math symbols, new environments and stuff
\usepackage{unicode-math} % for changing math font and unicode symbols
\usepackage[style=english]{csquotes} % fancy quoting
\usepackage{microtype} % for better font rendering
\usepackage{hyperref} % for refs and URLs
\usepackage{graphicx} % for images (and title page)
\usepackage{geometry} % for margins in title page
\usepackage{tabu} % for tabulars (and title page)
\usepackage[backend=bibtex, style=numeric-comp, sorting=none]{biblatex} % for bibliography
\usepackage{placeins} % for float barriers
\usepackage{titlesec} % for section break hooks
\usepackage{listings} % for listings 
\usepackage{upquote} % for good-looking quotes in source code (used for custom languages)
\usepackage{xcolor} % colors!
\usepackage{enumitem} % for unboxed description labels (long ones)
\usepackage{caption}

\lstloadlanguages{C,haskell,bash,java}

\lstset{
    frame=none,
    xleftmargin=2pt,
    stepnumber=1,
    numbers=left,
    numbersep=5pt,
    numberstyle=\ttfamily\tiny\color[gray]{0.3},
    belowcaptionskip=\bigskipamount,
    captionpos=b,
    escapeinside={*'}{'*},
    language=haskell,
    tabsize=2,
    emphstyle={\bf},
    commentstyle=\it,
    stringstyle=\mdseries\rmfamily,
    showspaces=false,
    keywordstyle=\bfseries\rmfamily,
    columns=flexible,
    basicstyle=\small\sffamily,
    showstringspaces=false,
    morecomment=[l]\%,
    breaklines=true
}
\renewcommand\lstlistingname{Листинг}

\defaultfontfeatures{Mapping=tex-text} % for converting "--" and "---"
\setmainfont{CMU Serif}
\setsansfont{CMU Sans Serif}
\setmonofont{CMU Typewriter Text}
\setmathfont{XITS Math}
\MakeOuterQuote{"} % enable auto-quotation

% new page and barrier after section, also phantom section after clearpage for
% hyperref to get right page.
% clearpage also outputs all active floats:
\newcommand{\sectionbreak}{\clearpage\phantomsection}
\newcommand{\subsectionbreak}{\FloatBarrier}
\newcommand\numberthis{\addtocounter{equation}{1}\tag{\theequation}}
\renewcommand{\thesection}{\arabic{section}} % no chapters
\numberwithin{equation}{section}
%\usetikzlibrary{shapes,arrows,trees}

\newcommand{\icode}[1]{\texttt{#1}} % inline code

% 20 - 25 стр достаточно

\bibliography{reference_list}

\begin{document}

% \thispagestyle{fancy}

\fancyhead[C]{
    Федеральное государственное бюджетное 
    образовательное учреждение высшего 
    профессионального образования\\
    <<Московский государственный технический 
    университет им.~Н.\,Э.~Баумана>>
}
\fancyfoot[C]{ г.\,Москва, 2013г. }

\vspace*{1cm}

\begin{flushright}
    \Large{ Факультет: }\\
    \large{ <<Информатика и системы управления>> }\\
    \Large{ Кафедра: }\\
    \large{ <<Программное обеспечение ЭВМ и\\ 
        информационные технологии>> }
\end{flushright}

\vspace{1cm}

\begin{LARGE} 
    \begin{center} 
        \begin{Large}
            Расчетно\,--\,пояснительная записка\\
            к курсовому проекту по курсу\\
            <<Конструирование компиляторов>>\\
        \end{Large}
        \vspace{2cm}
        <<Компилятор языка MiniML>>
    \end{center}
\end{LARGE}

\vspace{4cm}

\begin{flushright}
    \begin{tabular}{ll}
    Руководитель курсового проекта:&Просуков~Е.\,А.\\
    Исполнитель курсового проекта:&Гаврилов~П.\,Ъ.\\
                                  &Бережной~П.\,Ю.
    \end{tabular}
\end{flushright}

\newpage
\setcounter{page}{1}

\tableofcontents

\section{Введение}
В настоящей работе в рамках курса <<конструирование компиляторов>>
реализуется компилятор MiniML.

В первой части записки приводятся
использованная грамматика языка, описания лексического,
синтаксического, семантического анализаторов. Кратко описан поиск
лексических, синтаксических и семантических ошибок, таких как
несогласованность типов.

Вторая часть записки подробно описывает процесс генерации кода по построенному АСТ дереву.
Результатом курсовой работы является ПО,
которое при удачном лексическом, синтаксическом и семантическом
разборах создаёт абстрактное синтаксическое дерево, соответствующее
тексту исходной программы, преобразует полученное дерево в последовательность инструкций
и исполняет их.

Рассмотрим каждый указанный этап более подробно.

\section{Выбор платформы}
Для реализации фронтенда компилятора было решено использова язык
Haskell. Такое решение было принято из-за следующих преимуществ языка:
\begin{enumerate}
\item статическая, сильная типизация;
\item автоматический вывод типов, основанный на алгоритме Хиндли -- Милнера;
\item ленивые вычисления;
\item удобство написания парсеров.
\end{enumerate}
Был использован компилятор GHC 7.10.1 \cite{ghc}, система сборки и управления
пакетами и библиотеками языка Haskell Cabal 1.22.4.0 \cite{cabal}.

Для написания лексического и синтаксического анализаторов использовалась
библиотека parsec \cite{parsec}. Для сериализации AST в JSON-формат
использовалась библиотека aeson \cite{aeson}.

В качестве виртуальной машины для исполнения сгенерированных инструкций была
выбрана LLVM \cite{llvm}. Для генерирования инструкций использовалась библиотека
llvm-general \cite{llvm-general}.

\section{Описание языка}
Язык имеет следующие конструкции:
\begin{enumerate}
\item Беззнаковые целые числа и определённые над ними операции: +, - и *.
\item Логический тип (true, false), условные выражения и сравнение
  целых чисел (только = и <).
\item Рекурсивные функции и применение функций. Выражение 
  fun f (x : t) : s is e обозначает функцию типа t -> s, в которой
  инструкциям в e доступен x.
\item Высокоуровневые объявления: let x = e. Локальных объявлений нет.
\end{enumerate}

\section{Лексический анализ}
\subsection{Входные и выходные типы данных}
На этапе лексического анализа происходит преобразование входной строки
в последовательность токенов. Токеном может быть: идентификатор, оператор,
разезервированное слово, константа типа unsigned int или bool.

\subsection{Обнаруживаемые ошибки}
Единственные ошибки, обнаруживаемые фронтендом компилятора на этапе
лексического анализа, вызываются наличием в исходном коде компилируемой
программы недостимых символов. К примеру, такую ошибку вызовет встреченный
анализатором в любом месте программы символ \%.


Ошибки на этапе лексического анализа являются критическими: при возникновении
ошибки работа фронтенда немедленно завершается выводом сообщения, содержащего
позицию ошибки и подсказку о возможных способах её исправления.

\begin{lstlisting}[ language=haskell
                  , caption=Сообщение о лексической ошибке
                  , label=lst:lexical_error]
MiniML> let asd%dsa = 2
(line 1, column 8):
unexpected "%"
expecting letter or "="MiniML> let asd%dsa = 2
(line 1, column 8):
unexpected "%"
expecting letter or "="
\end{lstlisting}

Пример сообщения об ошибке, вызываемой наличием символа, который не
допускает грамматика языка, в исходном коде программы \ref{lst:lexical_error}.

\subsection{Реализация лексического анализатора}
Лексический анализатор был реализован с помощью библиотеки parsec,
которая позволяет с помощью комбинаторов из простых парсеров собирать
более сложные.

Исходный код лексического анализатора представлен в приложении 1.

\section{Синтаксический анализ}
\subsection{Входные и выходные типы данных}
На этапе синтаксического анализа происходит преобразование
последовательности токенов, сгенерированной лексическим анализатором,
в абстрактное синтаксическое дерево.


Описание выходные типов парсера представлено в листинге \ref{lst:parser}. 
Корнем дерева является команда высшего уровня ToplevelCmd.

\begin{lstlisting}[ language=haskell
                  , caption=Выходные типы данных парсера
                  , label=lst:parser]
-- Variable names
type Name = String

-- Types
data Ty = TInt         -- integers
        | TBool        -- booleans
        | TArrow Ty Ty -- functions

-- Expressions
data Expr = Var Name
          | Int Integer
          | Bool Bool
          | Times Expr Expr
          | Plus Expr Expr
          | Minus Expr Expr
          | Equal Expr Expr
          | Less Expr Expr
          | If Expr Expr Expr
          | Fun Name Name Ty Ty Expr
          | Apply Expr Expr

-- Toplevel commands
data ToplevelCmd = Expr Expr
                 | Def Name Expr
\end{lstlisting}

\subsection{Обнаруживаемые ошибки}
На данном этапе обнаруживаются ошибки несоответствия исходного кода
грамматике языка. Ошибки данного типа также являются критическими,
хотя более сложные парсеры могут восстанавливаться после ошибок.


При возникновении ошибки работа фронтенда немедленно завершается
выводом сообщения об ошибке, содержащего позицию ошибки и подсказку о
возможных способах её устранения.

Примеры сообщений об ошибках представлены в листинге \ref{lst:syntax_errors}.

\begin{lstlisting}[ language=haskell
                  , caption=Сообщения о синтаксической ошибке
                  , label=lst:syntax_errors]
MiniML> let x = (
(line 1, column 10):
unexpected end of input
expecting "(", "fun", natural, identifier, "if", "true" or "false"

MiniML> ((23 + 3) 
(line 1, column 10):
unexpected end of input
expecting ")"

MiniML> let a = + 2
(line 1, column 9):
unexpected "+"
expecting "(", "fun", natural, identifier, "if", "true" or "false"
\end{lstlisting}

\subsection{Реализация парсера}
Парсер был реализован с помощью библиотеки parsec. Реализация находится
в приложении 2.

\section{Семантический анализ}
\subsection{Вывод типов}
Вывод типов позволяет компилятору узнать тип значения выражения, не
указывая его явно. Например, ясно, что выражение e1 + e2 вычисляется
в тип int, а его операнды e1 и e2 также должны быть типа int.

Абстрактное синтаксическое дерево, полученное на выходе синтакического
анализатора проверяет модулем TypeCheck.hs на согласованность типов.

Исходный код модуля представлен в приложении 3.

\subsection{Обнаруживаемые ошибки}
Примеры обнаружение ошибок несогласованности типов представлены
в листинге \ref{lst:type_check_errors}.

\begin{lstlisting}[ language=haskell
                  , caption=Сообщения о несогласованности типов
                  , label=lst:type_check_errors]
MiniML> let a = true
a : bool
MiniML> a + 2
miniml: TE "Var "a" has type bool but is used as if it has type int"

MiniML> if 1 then false else true
miniml: TE "Int 1 has type int but is used as if it has type bool"
\end{lstlisting}

\subsection{Реализация семантического анализатора}
Функция typeOf, принимающая в качестве параметров контекст,
содержащий переменные и их типы, и выражение, возвращает тип
переданного выражения. Она рекурсивно обходит дерево, проверяя
согласованность типов во всех узлах переданного выражения.

\section{Генерация кода}
Основной задачей кодогенератора является преобразование
переданного ему Абстрактного Синтаксического Дерева в
набор машинных инструкций, которые в дальнейшем можно интерепретировать
тем или иным способом.

Рассмотрим более подробно методы, которые были использованы при написании генератора кода.

\subsection{Low Level Virtual Machine}

LLVM -- это универсальная система анализа, трансформации и оптимизации программ или, как её называют разработчики, <<compiler
infrastucture>>.

В основе LLVM лежит промежуточное представление кода (intermediate representation, IR), над которым можно производить
трансформации во время компиляции, компоновки (linking) и выполнения. Из этого представления генерируется оптимизированный
машинный код для целого ряда платформ, как статически, так и динамически (JIT-компиляция). LLVM поддерживает генерацию кода для
x86, x86-64, ARM, PowerPC, SPARC, MIPS, IA-64, Alpha.

\subsubsection{Типы данных}

В LLVM поддерживаются следующие примитивные типы:
\begin{itemize}
    \item Целые числа произвольной разрядности:
        \begin{itemize}
            \item \icode{i1} -- булево значение — 0 или 1
            \item \icode{i32} -- 32х разрядное целое
            \item \icode{i17} -- даже так
            \item \icode{i256} -- и так
        \end{itemize}
    \item Числа с плавающей точкой
    \item void — пустое значение
\end{itemize}

Кроме того поддержимаются производные типы, как например
\begin{itemize}
    \item \icode{i32*} -- указатель на целое число;
    \item \icode{[8 x double]} -- массив из восьми чисел с плавающей точкой;
    \item \icode{\{ i1 i10 i32 \}} -- структура из 3х элементов;
    \item \icode{i32 (i32, i32)} -- функции;
\end{itemize}

Система типов рекурсивна, поэтому можно использовать многомерные массивы, массивы структур, указатели на структуры и функции, и т.
д.

В виду специфики языка MiniML переменные ограничиваются
только двумя типами: 32х разрядное целое знаковое число (\icode{i32}) и
булево значение (\icode{i1}).

\subsubsection{Операции}

Большинство инструкций в LLVM принимают два аргумента (операнда) и возвращают одно значение (трёхадресный код). Значения
определяются текстовым идентификатором. Локальные значения обозначаются префиксом \icode{\%}, а глобальные — \icode{@}. Локальные значения также
называют регистрами, а LLVM — виртуальной машиной с бесконечным числом регистров. Пример
представлен в листинге \ref{lst:llvm_bytecode_example}.

\lstset{escapeinside={(*@}{@*)}}
\begin{lstlisting}[caption=Пример сгенерированного LLVM байткода
                  , label=lst:llvm_bytecode_example]
; ModuleID = 'MiniML'

; factorial function
define i32 @f(i32 %n) #0 {
entry:
  %0 = icmp eq i32 %n, 0
  br i1 %0, label %if.exit, label %if.else

if.else:                                          ; preds = %entry, %if.else
  ; %z = sum i32 %x, %y (*@\label{lst:comment_z1}@*)
  ; %z = sum i32 %z, 5 (*@\label{lst:comment_z2}@*)
  %n.tr2 = phi i32 [ %1, %if.else ], [ %n, %entry ]
  %accumulator.tr1 = phi i32 [ %2, %if.else ], [ 1, %entry ] (*@\label{lst:phi_fun}@*)
  %1 = add i32 %n.tr2, -1
  %2 = mul i32 %accumulator.tr1, %n.tr2
  %3 = icmp eq i32 %1, 0
  br i1 %3, label %if.exit, label %if.else

if.exit:                                          ; preds = %if.else, %entry
  %accumulator.tr.lcssa = phi i32 [ 1, %entry ], [ %2, %if.else ]
  ret i32 %accumulator.tr.lcssa
}

; Function Attrs: nounwind readnone
define i32 @fact(i32 %n) #0 {
entry:
  %0 = tail call i32 @f(i32 %n)
  ret i32 %0
}

; Function Attrs: nounwind readnone
define i32 @main() #0 { (*@\label{lst:main_fun}@*)
entry:
  %0 = tail call i32 @fact(i32 10)
  ret i32 %0
}
\end{lstlisting}

Тип операндов всегда указывается явно, и однозначно определяет тип результата.
Операнды арифметических инструкций должны иметь одинаковый тип, но сами инструкции <<перегружены>> для любых числовых типов и векторов.

Рассмотри более подробно особенности интерпретируемого LLVM байткода.

\subsubsection{Static Single Assignment form}

Static Single Assignment form (далее SSA) -- это такая форма промежуточного представления кода,
в которой любое значение присваивается только один раз. Таким образом, последовательность инструкций
в строках \ref{lst:comment_z1} -- \ref{lst:comment_z2} листинга \ref{lst:llvm_bytecode_example} неверна
и привидет к ошибкам компиляции. Новое значение переменной обязательно долно принять новое имя.

Код в SSA-форме удобно рассматривать не как линейную последовательность инструкций,
а как граф потока управления (control flow graph, CFG). Вершины этого графа -- так
называемые базовые блоки (basic blocks), содержащие последовательность инструкций,
заканчивающуюся инструкцией-терминатором, явно передающей управление в другой блок.
Базовые блоки в LLVM обозначаются метками, а терминаторами являются следующие инструкции:
\begin{itemize}
    \item \icode{ret тип значение} -- возврат значения из функции;
    \item \icode{br i1 условие, label метка\_1, label метка\_2} — условный переход;
    \item \icode{switch} -- обобщение br, позволяет организовать таблицу переходов:\\
        \icode{switch i32 \%n, label \%Default, [i32 0, label \%IfZero i32 5, label \%IfFive]};
    \item \icode{invoke} и \icode{unwind} -- используются для организации исключений;
    \item \icode{unreachable} -- специальная инструкция, показывающая компилятору, что выполнение никогда не достигнет этой точки.
    \item инструкция $\varphi$ -- специальная инструкция, которая возвращает одно из перечисленных значений в зависимости от того,
        какой блок передал управление текущему.
\end{itemize}

В LLVM функции $\varphi$ соответствует инструкция $\phi$, которая имеет следующую форму:
\begin{verbatim}
phi тип, [значение_1, label метка_1], ..., [значение_N, label метка_N]
\end{verbatim}

Пример использования этой функции можно увидеть в строке \ref{lst:phi_fun} листинга \ref{lst:llvm_bytecode_example},
где приводится код функции вычисления факториала числа.

\subsubsection{Память}

Помимо значений-регистров, в LLVM есть и работа с памятью. Значения в памяти адресуются типизированными указателями.
Обратиться к памяти можно только с помощью двух инструкций, названия которых говорят сами за себя: \icode{load} и \icode{store}.

Но чтобы пользоваться указателями, надо как-то выделять память под значения, на которые они указывают.
Инструкция \icode{malloc} транслируется в вызов одноименной системной функции и выделяет память на куче, возвращая значение -- указатель
определенного типа. В паре с ней идёт инструкция \icode{free}.

Для выделения памяти на стеке служит функции \icode{alloca}.
Память, выделенная этой функцией автоматически освобождается при выходе из функции при помощи инструкций \icode{ret} или \icode{unwind}.

\subsubsection{Оптимизации}

LLVM имеет встроенный оптимизатор кода, которые способен выполнять следующие оптимизации кода:
\begin{itemize}
    \item Удаление неиспользуемого кода (dead code elimination).
    \item Выделение одинаковых подвыражений (common subexpression elimination).
    \item Распространение констант (constant propagation, condition propagation).
    \item Инлайн-подстановка функций.
    \item Раскрутка и размыкание циклов, вынос инвариантов за пределы цикла.
    \item Разворот хвостовой рекурсии.
\end{itemize}

Преобразование может быть не только оптимизирующим, но и использоваться для анализа и инструментации. Например, LLVM может
генерировать CFG в формате Graphviz \cite{graphviz}.

\subsubsection{JIT компиляция}

Для вывода результата выражения была использована технология Just-in-time compilation (JIT компиляция) --
технология увеличения производительности программных систем, использующих байт-код, путём компиляции байт-кода в машинный код или
в другой формат непосредственно во время работы программы.

В случае, если требуется вычислить выражение любой сложности,
формируется функция \icode{main} (строка \ref{lst:main_fun}, листинг \ref{lst:llvm_bytecode_example}),
которая подается вместе с остальным кодом компилятору.

В результате значением исходного выражения окажется значение, которые вернула функция \icode{main}.
Более подробно ознакомиться с кодом компиляции LLVM байткода можно в приложении 6.

\subsection{Особенности реализации}

При реализации генератора кода было встретилось множество проблем, связанных с особенностями языка
MiniML.
Рассмотрим некоторые из них.

\subsubsection{Глобальные переменные}

Рассмотрим в листинге \ref{lst:global_var_miniml} пример кода на языке MiniML.
\begin{lstlisting}[caption=Создание глабальной переменной
                  , label=lst:global_var_miniml]
MiniML> let a = 2 + 3 + fact(4) * 3 - 456
MiniML> a
- : int = -379
\end{lstlisting}

Данное в листинге \ref{lst:global_var_miniml} выражение безусловно является сложным. При его вычислении требуется вызвать рекурсивную
функцию вычисления факториала числа, и провести некоторые математические операции с результатом.

Данное выражение требует дважды вызвать интерпретатор LLVM байткода.
Первый раз это необходимо для того, чтобы вычислить правую часть выражения,
а во второй раз -- чтобы создать глобальную переменную \icode{@a} и присвоить ей значение,
вычисленное при первом проходе.

В виду простоты языка MiniML было решено избежать повторного вызова интерпретатора.
Данное решение отличается от предыдущего более простой, но более ресурсоемкой реализацией.

Рассмотрим более подробно схему, используемую в данном случае.

\begin{lstlisting}[caption=Пример сгенерированного LLVM байткода для глобальной переменной
                  , label=lst:global_var]
define i32 @a_global(i1 %_) #0 {
entry:
  %0 = tail call i32 @fact(i32 4)
  %1 = mul i32 %0, -3
  %2 = add i32 %1, -451
  ret i32 %2
}

define i32 @main() #0 {
entry:
  %0 = tail call i32 @a_global(i1 undef)
  ret i32 %0
}
\end{lstlisting}

В листинге \ref{lst:global_var} представлен оптимизированный код для выражения из листинга \ref{lst:global_var_miniml}.
Видно, что вместо создания глобальной переменной \icode{@a} была создана функция \icode{@a\_global},
которая вызывается всякий раз при обращении к переменной.

\begin{lstlisting}[ language=haskell
                  , caption=Реализация глобалой переменной в программе
                  , label=lst:lexical_error]
codegenTop :: TC.Ctx -> Vars -> S.ToplevelCmd -> LLVM ()
codegenTop _ globVars (S.Def var_name (S.Fun name argname argtype rettype body)) = do
  define (toType rettype) name fnargs bls
  define (toType rettype) var_name var_args var_bls
  modify $ \s -> s { ty = rettype }
  where
    (fnargs, bls) = genFun globVars (S.Fun name argname argtype rettype body)
    (var_args, var_bls) = genFun globVars (S.Fun var_name argname argtype rettype pseudo_body)
    pseudo_body = (S.Apply (S.Var name) (S.Var argname))
\end{lstlisting}

С точки зрения вычислений это однозначно является не самым удачным решением, однако язык MiniML использован
исключительно в демонстрационных целях и данное решение вполне является оправданным.

Более подробно с кодом можно ознакомиться в приложениях 4 и 5.

\subsubsection{Переменные, хронящие ссылки на функции}

Аналогичным способу, указанному в листинге \ref{lst:global_var}, образом была
реализована геренация глобальных переменных, ссылающихся на функции.

\begin{lstlisting}[caption=Пример сгенерированного LLVM байткода для глобальной переменной
                  , label=lst:func_var]
MiniML> let b = fun f (x:bool):int is if x then 10 else 20

define i32 @f(i1 %x) #0 {
entry:
  %. = select i1 %x, i32 10, i32 20
  ret i32 %.
}

define i32 @b(i1 %x) #0 {
entry:
  %0 = tail call i32 @f(i1 %x)
  ret i32 %0
}
\end{lstlisting}

Из листинга \ref{lst:func_var} видно, что вместо того, чтобы сделать одну функцию \icode{@f} и
переменную, которая хранить ссылку на функцию,
было принято решение сделать 2 функции (\icode{@f} и \icode{@b}), одна из которых вызывает другую.

Таким образом, при обращении к переменной \icode{b} в действительности приосходит вызов
функции \icode{@b}, которая в свою очередь передает управление функции \icode{@f}.

\subsubsection{Перекрытие переменных}

Для упрощения логики построения байт кода перекрытие переменных было реализовано следующим образом.
Вместо того, чтобы хранить весь код в виде последовательности АСТ деревьев и при каждом
вызове интерпретатора снова и снова генерировать код занова, в памяти хранится
лишь список глобальных переменных и уже частично сгенерированный код.

\begin{lstlisting}[caption=Пример перекрытия переменной \icode{a} новым значением
                  , label=lst:again_vars]
MiniML> let a = 1

define i32 @a_global(i1 %_) #0 {
entry:
  ret i32 1
}

MiniML> let a = 2

define i32 @a_global(i1 %_) #0 {
entry:
  ret i32 1
}

define i32 @a_2_global(i1 %_) #0 {
entry:
  ret i32 2
}

\end{lstlisting}

Таким образом, при попытке перекрыть переменную, вместо того, чтобы
убрать из генерируемого кода старую переменную (если точнее, функцию)
проиходит лишь создание новой функции с новым именем. Старая функция останется в памяти, однако при исполнении
она будет удалена из конечной программы на проходе оптимизатора (листинг \ref{lst:again_vars}).

При обращении к переменной во второй раз уже будет вызвана новая функция с новым значением.


\section{Вывод}
В рамках данной курсовой работы был реализован платформонезависимый компилятор
языка MiniML. Были реализованы лексический, синтаксический и
семантический анализаторы. Результирующее абстрактное синтаксическое
дерево представляется в формате JSON. Сгенерированное ACT дерево преобразуется
в байткод LLVM и исполняется на лету.

\section{Список литературы}
\printbibliography[heading=none]

\section*{\titleline[r]{Приложение 1}}
\addcontentsline{toc}{section}{Приложение 1}
\subsection*{Исходный код лексического анализатора}
\lstinputlisting{../src/Lexer.hs}
\clearpage

\section*{\titleline[r]{Приложение 2}}
\addcontentsline{toc}{section}{Приложение 2}
\subsection*{Исходный код синтаксического анализатора}
\lstinputlisting{../src/Parser.hs}
\clearpage

\section*{\titleline[r]{Приложение 3}}
\addcontentsline{toc}{section}{Приложение 3}
\subsection*{Исходный код семантического анализатора}
\lstinputlisting{../src/TypeCheck.hs}
\clearpage

\section*{\titleline[r]{Приложение 4}}
\addcontentsline{toc}{section}{Приложение 4}
\subsection*{Исходный код кодогенератора}
\lstinputlisting{../src/Codegen.hs}
\clearpage

\section*{\titleline[r]{Приложение 5}}
\addcontentsline{toc}{section}{Приложение 5}
\subsection*{Исходный код логики управления кодогенерацией}
\lstinputlisting{../src/Emit.hs}
\clearpage

\section*{\titleline[r]{Приложение 6}}
\addcontentsline{toc}{section}{Приложение 6}
\subsection*{Исходный код компилятора байткода}
\lstinputlisting{../src/JIT.hs}
\clearpage

\section*{\titleline[r]{Приложение 7}}
\addcontentsline{toc}{section}{Приложение 7}
\subsection*{Исходный код основного модуля}
\lstinputlisting{../src/Main.hs}
\clearpage

\end{document}
