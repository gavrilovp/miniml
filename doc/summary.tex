\documentclass[a4paper,12pt]{article}

% polyglossia should go first!
\usepackage{polyglossia} % multi-language support
\setmainlanguage{russian}
\setotherlanguage{english}

\usepackage{amsmath} % math symbols, new environments and stuff
\usepackage{unicode-math} % for changing math font and unicode symbols
\usepackage[style=english]{csquotes} % fancy quoting
\usepackage{microtype} % for better font rendering
\usepackage{hyperref} % for refs and URLs
\usepackage{graphicx} % for images (and title page)
\usepackage{geometry} % for margins in title page
\usepackage{tabu} % for tabulars (and title page)
\usepackage[backend=bibtex, style=numeric-comp, sorting=none]{biblatex} % for bibliography
\usepackage{placeins} % for float barriers
\usepackage{titlesec} % for section break hooks
\usepackage{listings} % for listings 
\usepackage{upquote} % for good-looking quotes in source code (used for custom languages)
\usepackage{xcolor} % colors!
\usepackage{enumitem} % for unboxed description labels (long ones)
\usepackage{caption}

\defaultfontfeatures{Mapping=tex-text} % for converting "--" and "---"
\setmainfont{CMU Serif}
\setsansfont{CMU Sans Serif}
\setmonofont{CMU Typewriter Text}
\setmathfont{XITS Math}
\MakeOuterQuote{"} % enable auto-quotation

% new page and barrier after section, also phantom section after clearpage for
% hyperref to get right page.
% clearpage also outputs all active floats:
\newcommand{\sectionbreak}{\clearpage\phantomsection}
\newcommand{\subsectionbreak}{\FloatBarrier}
\newcommand\numberthis{\addtocounter{equation}{1}\tag{\theequation}}
\renewcommand{\thesection}{\arabic{section}} % no chapters
\numberwithin{equation}{section}
%\usetikzlibrary{shapes,arrows,trees}

% 20 - 25 стр достаточно

\bibliography{reference_list}

\begin{document}

\tableofcontents

\section{Введение}
В настоящей работе в рамках курса <<конструирование компиляторов>>
реализуется фронтенд компилятора MiniML. В записке приводятся
использованная грамматика языка, описания лексического,
синтаксического, семантического анализаторов. Кратко описан поиск
лексических, синтаксических и семантических ошибок, таких как
несогласованность типов. Результатом курсовой работы является ПО,
которое при удачном лексическом, синтаксическом и семантическом
разборах создаёт абстрактное синтаксическое дерево, соответствующее
тексту исходной программы.

\section{Выбор платформы} 
Для реализации фронтенда компилятора было решено использова язык
Haskell. Такое решение было принято из-за следующих преимуществ языка:
\begin{enumerate}
\item статическая, сильная типизация;
\item автоматический вывод типов, основанный на алгоритме Хиндли -- Милнера;
\item ленивые вычисления;
\item удобство написания парсеров.
\end{enumerate}
Был использован компилятор GHC 7.10.1 \cite{ghc}, система сборки и управления
пакетами и библиотеками языка Haskell Cabal 1.22.4.0 \cite{cabal}.


Для написания лексического и синтаксического анализаторов использовалась
библиотека parsec \cite{parsec}. Для сериализации AST в JSON-формат
использовалась библиотека aeson \cite{aeson}.

\section{Описание языка}
Описание языка.

\section{Список литературы}
\printbibliography[heading=none]

\end{document}